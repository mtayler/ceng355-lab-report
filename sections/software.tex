\subsection{Software Implementation}
The main program called functions provided by \lstinline{analog.h} and
\lstinline{lcd.h} to sequentially initialize each component. Components
were initialized in the order of: the \gls{adc}, the \gls{dac}, the
\gls{lcd}, and the frequency monitor. Details of registers and other
technical notes are available in the reference manual published by
STMicroelectronic's \cite{ref}.

\lst{Initialization order}{code:init-order}{42}{57}{src/main.c}

\subsubsection{ADC Initialization}
Initialization of the gls{adc} requires initialization of the \gls{gpio}
port C interface (to control the POT\_EN signal of the Project Board),
GPIOC (to read the potentiometer value), and the gls{adc}. The C code in
Listing~\ref{code:adc-gpioc-conf} comprises the initialization of the
GPIOC register for the gls{adc}.

\lst{GPIO configuration for ADC}{code:adc-gpioc-conf}{35}{48}{src/analog.c}

Firstly, the GPIOC clock is ensured to be running, followed by
configuration of the pins. The pins are put in a push-pull output
configuration at the highest speed, without any pull-up or pull-down
resistors.

\lst{ADC configuration}{code:adc-conf}{50}{74}{src/analog.c}

Next the \gls{adc} proper is initialized: the C code in
Listing~\ref{code:adc-conf} accomplishes this.  The \gls{hsi} clock, which
supplies the \gls{adc}'s clock, is enabled and the \gls{adc} started in preparation
for configuration. Lines~57-68 configure the gls{adc}: setting the clock as
the dedicated (\gls{hsi}) clock, selecting input channel 0 (corresponding to
parallel pin A0), continuous conversion mode is enabled to continuously
provide the digitized value of the PA0 pin. Finally, the function
triggers conversion to start after waiting until the gls{adc} reports that it
has stabilized.

\subsubsection{DAC Initialization}
Enabling the \gls{dac} requires configuration of the PA4 pin and
initialization of the \gls{dac} clock. Listing~\ref{code:dac-init} shows
the C code initializing the \gls{dac}.

\lst{DAC configuration}{code:dac-init}{78}{92}{src/analog.c}

The mode of PA4 is set to an open-drain analog output, allowing a
variable voltage to be placed on PA4, with pull-up and pull-down
pins disabled, at the highest speed setting to ensure the pin updates
quickly. Next, the \gls{dac}'s clock is enabled and the \gls{dac}
enabled by writing the \gls{dac}'s \gls{cr}'s enable bit.

\subsubsection{LCD Initialization}

To initialize the \gls{lcd}, the \gls{spi}, GPIOB, and GPIOC clocks are
all initialized (Listing~\ref{code:lcd-init-clocks}).

\lst{SPI, GPIOB, GPIOC clock enable}{code:lcd-init-clocks}{15}{21}{src/lcd.c}

The \gls{lck} pin (PC2) of the \gls{spi} shift-register is set to output
mode, and the \gls{mosi} (PB5) and \gls{sck} (PB3) \gls{spi} pins are
set to alternate function mode (Listing~\ref{code:lcd-init-output}).

\lst{SPI pin mode configuration}{code:lcd-init-output}{23}{30}{src/lcd.c}

Each \gls{spi} related pin is set to push-pull mode with pull-up and
pull-down resistors disabled and high-speed mode
(Listing~\ref{code:lcd-init-output-conf}).

\lst{SPI pin output configuration}{code:lcd-init-output-conf}{32}{45}{src/lcd.c}

After configuring the output pins, \gls{tim3} is configured, allowing
\gls{spi} writes to be delayed and the \gls{lcd} time to complete
the commands sent:

\lst{TIM3 configuration}{code:lcd-tim3-conf}{47}{62}{src/lcd.c}

The clock for \gls{tim3} is enabled, then the clock is configured to use
buffered auto-reload, count down from the set \lstinline{CNT} register
value, stop in the event of an overflow (which should never occur), and
only interrupt in the event of an overflow not when the counter reaches
0. This configuration allows a value to be written to the timer which
then counts down to 0, and the timer can be polled until the count is
low enough (and enough time has passed), then allowing another command to be
sent. This avoids the potential for an overflow if the counter was
initialized to zero and counted up to or past the desired value. \\

The prescaler is set to 0, the auto-reload delay set to the lowest
possible, writing to the \lstinline{EGR} register is a carry-over from
other timer initializations and is not necessary in this case; then, the
\lstinline{MAX_DELAY} value (the maximum time the LCD will take to
execute a command) is loaded into the timer's \lstinline{CNT}
register, and the timer finally started. This timer is utilized
by checking the \lstinline{CNT} value rather than the updates generated
for the purpose of allowing a potentially shorter delay. In the end this
potential for a shorter delay was not utilized, however checking the
\lstinline{CNT} register rather than an update bit was retained.

Next, in Listing~\ref{code:spi-init}, \gls{spi} is configured and
enabled with: unidirectional transmit, master mode, a data-size of 8
bits, \gls{cpol} and \gls{cpha} are set for a falling edge clock pulse, software chip select, a 256 baud-rate
prescaler, \gls{msb} first output, and a 7 bit \gls{crc} polynomial.

\lst{SPI initialization}{code:spi-init}{64}{78}{src/lcd.c}

Finally, in Listing~\ref{code:lcd-init}, the command interface and
the display of the \gls{lcd} are initialized: the \gls{lcd} is set to a
4-bit interface with 2 lines, followed by the LCD being cleared and
homed with display shift disabled and cursor movement direction set to
the right, cursor blinking being disabled. The persistent, unchanging
characters of the \gls{ui} are then written to the \gls{lcd}.

\lst{LCD initialization}{code:lcd-init}{80}{114}{src/lcd.c}

\subsubsection{Frequency Monitor Initialization}

Initialization of the frequency monitor required the GPIOA and
\gls{tim2} clocks to be enabled, GPIOA pin 1 configured as an input,
and configuration of \gls{tim2}. \\

PA1 is first configured with the C code in Listing~\ref{code:fm-init-gpio}.

\lst{GPIOA pin 1 configuration}{code:fm-init-gpio}{94}{100}{src/analog.c}

The GPIOA clock is first ensured to be enabled, followed by the setting
pin PA1 to input mode with pull-up and pull-down resistors disabled.
\gls{tim2} is then initialized, seen in Listing~\ref{code:fm-init-tim2}.

\lst{TIM2 initialization}{code:fm-init-tim2}{102}{115}{src/analog.c}

\gls{tim2} is configured with a buffered auto-reload in count up mode,
with stop on overflow set and updates events enabled with interrupts
only enabled for overflow. This allows the timer to notify in the event
of an iInterrupt and avoid erroneous frequencies measurements when the
frequency drops below a corresponding period of approximately
\SI{90}{s}. \\

Next, the interrupt routines are configured through the \gls{nvic} as
seen in Listing~\ref{code:fm-init-nvic}.

\lst{NVIC configuration}{code:fm-init-nvic}{117}{137}{src/analog.c}

The priority of the interrupt is set to 0, as it is one of only two
interrupts that should be firing in this system. \gls{tim2} interrupts
are enabled in the \gls{nvic} to allow notifiication of overflow, and
interrupt generation is enabled in \gls{tim2}. The external interrupt
line carrying the external waveform, \gls{exti1}, is mapped to PA1. The
external interrupt line has interrupts enabled with a rising-edge
trigger and interrupts for \gls{exti1} unmasked in the \gls{nvic} so
they are processed. Finally, \gls{exti1} is configured to have a
priority  of 0 in the \gls{nvic}, and enabled.

\subsubsection{LCD Implementation} \label{sec:lcd-imp}
Utilization of the LCD was accomplished through the
\lstinline{void lcd_char(char c)} and
\lstinline{void lcd_cmd(uint8_t data)} functions. \lstinline{lcd_char}
provides an interface to specify a character which is then translated to
the proper LCD command and displayed at the current cursor position.
\lstinline{lcd_cmd} does not translate it's input, instead implementing
the transmission of 8 bits of data to the LCD over the 4 bit interface.
Definitions of \lstinline{lcd_char} and \lstinline{lcd_cmd} can be seen
in Listing~\ref{code:lcd-functions}

\lst{LCD interface function definitions}{code:lcd-functions}{116}{163}{src/lcd.c}

Both commands split the 8 bits of data to be sent into 4 bit packets to
be sent over the 4 bit interface: where 4 bits of the data sent are
control bits and 4 bits are data. The difference between the two
commands resides in the control bits, where \lstinline{lcd_char} has the
\gls{rs} bit LOW whereas \lstinline{lcd_cmd} send a HIGH \gls{rs} bit.
\\

Both functions utilize \lstinline{void spi_write(uint8_t data)}, a
function providing an abstracted interface to utilize the 74HC164
serial-in, parallel-out shift register which interfaces with the
\gls{lcd}. The command ensures suitable time has passed by polling
\gls{tim3}, where it then sets the \gls{lck} signal LOW, loads the data
into the shift register over \gls{spi} before finally setting the
\gls{lck} pin HIGH again and resets the \gls{tim3} delay counter.

Displaying frequency and resistance information on the \gls{lcd} is
accomplished by first setting up the \gls{ui} by writing the persistent
characters, where the underscores between represent spaces which are
updated with the read values:

\begin{lstlisting}
	F:____Hz
	R:____Oh
\end{lstlisting}

When the frequency monitor reads an updated value (see~\ref{sec:fm-imp}
for details) the LCD is updated with the new resistance and calculated
frequency values. When these values are received, the code in
Listing~\ref{code:lcd-value-update} updates the frequency of the \gls{ui}.

\lst{LCD value update}{code:lcd-value-update}{169}{174}{src/analog.c}

The value of the frequency is converted to an array of ASCII characters
by calling \lstinline{char* num_to_ascii(uint16_t)}
(Listing~\ref{code:num-to-ascii}) , which are then written to the LCD.
The loop decrements to read the array from end to start, as the way
\lstinline{num_to_ascii} operates stores the digits least significant
digit first. The same method is used to update the resistance value,
however the resistance value is rounded to the nearest 100 as the
frequency does not change until the resistance varies by approximately
this amount, and to improve readability by reducing flickering of the
digits.

\lst{Convert a value to ASCII characters}{code:num-to-ascii}{165}{184}{src/lcd.c}


\subsubsection{Frequency Monitor Implementation} \label{sec:fm-imp}

The frequency monitor's main components consist of \gls{tim2} and an
interrupt routine triggered by \gls{exti1}. Defined in
Listing~\ref{code:fm-imp}, \lstinline{void EXTI0_1_IRQHandler()} is the
routine called when the input PA1 sees a rising-edge.

\lst{Frequency monitor interrupt routines}{code:fm-imp}{139}{198}{src/analog.c}

Every $2n, n=0,1,\dots$  executions (executions 0,2,\dots where
\lstinline{first_edge=1}) of this routine:

\begin{enumerate}
	\item Interrupts are disabled.
	\item The interrupt request flag is checked.
	\item The count of \gls{tim2} to 0 and the timer started
	\item The interrupt flag is cleared and interrupts are re-enabled.
	\item The service routine returns.
\end{enumerate}

Every $2n+1, n=0,1,\dots$ executions (executions 1,3,\dots where
\lstinline{first_edge=0}) of this routine:

\begin{enumerate}
	\item Interrupts are disabled.
	\item The interrupt request flag is checked.
	\item The \gls{tim2} clock is stopped.
	\item The current count is stored in \lstinline{uint32_t count}.
	\item The frequency and resistance displayed on the \gls{lcd} is
		updated (see~\ref{sec:lcd-imp}).
	\item \gls{tim3} is utilized to delay updates to a reasonable period
		(to reduce flickering of the numbers on the \gls{lcd}).
	\item The interrupt flag is cleared and interrupts are re-enabled.
	\item The service routine returns.
\end{enumerate}

The end result is measurement of the time between 2 successive
rising-edges, giving the frequency of the input waveform. \\ 

The \lstinline{void TIM2_IRQHandler()} interrupt routine, also visible in
Listing~\ref{code:fm-imp}, is the routine called in the event of an
interrupt generated by the \gls{tim2}, which only occurs in the event of
an overflow.

