%  Explain your design methodology, such as your design steps, lab work
%  partitioning, special techniques used, etc. Clearly explain any
%  special programming considerations (e.g., explain how and why you set
%  up control registers, how and why you check for flags, etc.)

\section{Design Solution}
\subsection{Planning}
\subsubsection{Work Partitioning}
Based on experience from the first lab, and personal preference, lab
work was partitioned accordingly. The project was split into the
embedded programming and accessory circuit design/implementation, where
the majority of the embedded programming was assigned to Tayler Mulligan
and the accessory timer circuit to Raymond Bamford. The partitioning was
not strict, with members collaborating where necessary or convenient.

\subsubsection{Technique and Technologies}
Git and GithHub were utilized for the project to provide team access and
syncing between lab computers (see
\url{https://github.com/tamul/ceng355-lab-project}). The source code was
split into three files: \filename{main.c} (see \ref{app:main}), containing the main program;
\filename{analog.c} (see \ref{app:analog}), containing ADC, DAC, and frequency monitoring
code; and \filename{lcd.c} (see \ref{app:lcd}), containing code related to the LCD; each
with corresponding header files.

\subsection{Implementation}
The main program called functions provided by \lstinline{analog.h} and
\lstinline{lcd.h} to sequentially initialize each component. Components
were initialized in the order of: the ADC, the DAC, the LCD, and the
frequency monitor. 

\subsubsection{ADC Initialization}
Initialization of the ADC requires initialization of the GPIOC interface
(to control the POT\_EN signal of the Project Board), GPIOC (to read the
 potentiometer value), and the ADC. The C code in
 Listing~\ref{code:adc-gpioc-conf} comprises the initialization of the
 GPIOC register for the ADC.
\lstinputlisting[caption={GPIO configuration for ADC},label={code:adc-gpioc-conf},linerange={36-48},firstnumber=36]{src/analog.c}

Firstly, the GPIOC clock is ensured to be running, followed by
configuration of the pins. The pins are put in a push-pull output
configuration at the highest speed, without any pull-up or pull-down
resistors. The GPIOA clock is guaranteed to be running as the
communication between the computer and STM32 micro are over the parallel
A ports. The PA0 pin is set to ``analog'' mode. \\

\lstinputlisting[caption={ADC configuration},label={code:adc-conf},linerange={50-73},firstnumber=50]{src/analog.c}

Next the ADC proper is initialized: the C code in
Listing~\ref{code:adc-conf} accomplishes this.  The HSI14 clock, which
supplies the ADC's clock, is enabled and the ADC started in preparation
for configuration. Lines~57-68 configure the ADC: setting the clock as
the dedicated (HSI14) clock, selecting input channel 0 (corresponding to
parallel port A0), continuous conversion mode is enabled to continuously
provide the digitized value of the PA0 pin. Finally, the function
triggers conversion to start after waiting until the ADC reports that it
has stabilized.
