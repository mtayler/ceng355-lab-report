%  Explain your design methodology, such as your design steps, lab work
%  partitioning, special techniques used, etc. Clearly explain any
%  special programming considerations (e.g., explain how and why you set
%  up control registers, how and why you check for flags, etc.)

\section{Design Solution}
\subsection{Planning}
\subsubsection{Work Partitioning}
Based on experience from the first lab, and personal preference, lab
work was partitioned accordingly. The project was split into the
embedded programming and accessory circuit design/implementation, where
the majority of the embedded programming was assigned to Tayler Mulligan
and the accessory timer circuit to Raymond Bamford. The partitioning was
not strict, with members collaborating where necessary or convenient.

\subsubsection{Technique and Technologies}
Git and GithHub were utilized for the project to provide team access and
syncing between lab computers (see
\url{https://github.com/tamul/ceng355-lab-project}). The source code was
split into three files: \filename{main.c} (see \ref{app:main}), containing the main program;
\filename{analog.c} (see \ref{app:analog}), containing ADC, DAC, and frequency monitoring
code; and \filename{lcd.c} (see \ref{app:lcd}), containing code related to the LCD; each
with corresponding header files.
